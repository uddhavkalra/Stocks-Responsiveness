\documentclass[/Users/uddhav/Desktop/MA Thesis/Stocks-Responsiveness/LaTex/master_file.tex]{article}
\usepackage[a4paper, margin=0.75in]{geometry}
\usepackage{setspace}    

\usepackage{amsmath}
\usepackage{amssymb}
\usepackage{amsthm}          
\usepackage{mathtools}    
\usepackage{bm}             
\usepackage{dsfont}          
\usepackage{nicefrac}        

\usepackage{subfiles} 

\usepackage{indentfirst}
\usepackage[bottom]{footmisc}

\setstretch{1.5}

\begin{document}
Passive investing has grown significantly over the past decade, surpassing Active investing in equities. It is important to discuss what passive investing is beforehand, as it could be argued that no one is a passive investor as everyone makes the active decision to invest.

Passive investing has two primary definitions:
\begin{itemize}
\item[a.] Passive investors choose a portfolio, buy it and hold it long-term with no regard for profiting from short term variations or frequent trading. (Moltke \& Sl\o k, 2024)
\item[b.] A passive investor holds every security from the market, with each represented in the same manner as in the market. (Sharpe, 1991)
\end{itemize}
For the purposes of this paper, we will adopt Sharpe's definition. 

Recently Total Assets in Index Funds overtook Total Assets in Non-Index Funds. Figures 1 and 2 illustrate this development showing the level and share of assets.

Haddad, Huebner and Loualiche (2024), show through their model that an increase in the share of passive investing leads to lower price elasticities of demand which can lead to higher volatility, lower efficiency and illiquidity. If stocks become less responsive to trading demand, they may also adjust differently to new information.

This motivates the research question, ``How has passive investing affected stock responsiveness to news.''
\end{document}