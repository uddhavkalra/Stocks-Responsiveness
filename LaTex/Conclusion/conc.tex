\documentclass[/Users/uddhav/Desktop/MA Thesis/Stocks-Responsiveness/LaTex/master_file.tex]{article}
\usepackage[a4paper, margin=0.75in]{geometry}
\usepackage{setspace}    

\usepackage{amsmath}
\usepackage{amssymb}
\usepackage{amsthm}          
\usepackage{mathtools} 

\begin{document}

Since the creation of the first index funds and ETFs, passive investing has been growing consistently. As the size of assets under index funds has increased and overtaken assets under non-index funds, price informativeness has decreased.

This paper provides evidence that passive investing does lead to a decrease in price informativeness by using volatility as a proxy. These findings complement recent theoretical work that links the rise of passive investing to declining price elasticity of demand and reduced information efficiency (Haddad et al., 2024). By using an event-study approach focused on earnings announcements, this paper offers empirical evidence of this mechanism at work. Volatility for stocks with higher passive ownership is lower during the pre-announcement period, while volatility for those stocks is elevated during the post-announcement period.

Lower price informativeness leads to capital misallocation, but does allow active managers a potential trade strategy by using soft information in the pre-earnings period and options contracts expiring post-earnings. 

The drawback of this paper is that the measure of share passively held relies on funds to self identify as index funds and so misses funds that are in practice index funds but not in name. Future work could provide further robustness by employing an alternative measure using 13F filings to compute the elasticity of demand of stocks for each fund, which could be used alongside the measure employed in this paper.

In summary, this paper provides causal evidence that the growing dominance of passive investing has meaningful effects on how information is incorporated into stock prices. By employing an event study framework with firm fixed effects and two stage least squares estimation, we find that firms with higher passive ownership exhibit lower volatility before earnings announcements and heightened volatility afterwards, consistent with reduced price informativeness. These results underscore the role of passive investing in reshaping market dynamics and contribute to the broader debate on whether the shift toward index-based investing enhances or impairs market efficiency. The findings suggest that as passive ownership continues to expand, the information environment of public equities may become increasingly segmented, with important implications for investors, policy makers, and the efficiency of capital markets.

\end{document}